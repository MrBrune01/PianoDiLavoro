%----------------------------------------------------------------------------------------
%   USEFUL COMMANDS
%----------------------------------------------------------------------------------------

\newcommand{\dipartimento}{Dipartimento di Matematica ``Tullio Levi-Civita''}

%----------------------------------------------------------------------------------------
% 	USER DATA
%----------------------------------------------------------------------------------------

% Data di approvazione del piano da parte del tutor interno; nel formato GG Mese AAAA
% compilare inserendo al posto di GG 2 cifre per il giorno, e al posto di 
% AAAA 4 cifre per l'anno
\newcommand{\dataApprovazione}{Data}

% Dati dello Studente
\newcommand{\nomeStudente}{Mattia}
\newcommand{\cognomeStudente}{Brunello}
\newcommand{\matricolaStudente}{2009096}
\newcommand{\emailStudente}{mattia.brunello@studenti.unipd.it}
\newcommand{\telStudente}{+ 39 342 88 68 740}

% Dati del Tutor Aziendale
\newcommand{\nomeTutorAziendale}{Nicola}
\newcommand{\cognomeTutorAziendale}{Marchesan}
\newcommand{\emailTutorAziendale}{nicola.marchesan@admaiorastudio.com}
\newcommand{\telTutorAziendale}{+ 39 333 49 19 527}
\newcommand{\ruoloTutorAziendale}{CEO}

% Dati dell'Azienda
\newcommand{\ragioneSocAzienda}{AdMaioraStudio}
\newcommand{\indirizzoAzienda}{P.zza Serenissima 40, int 101, Castelfranco V. (TV)}
\newcommand{\sitoAzienda}{www.admaiorastudio.com}

% Dati del Tutor Interno (Docente)
\newcommand{\titoloTutorInterno}{Prof.ssa}
\newcommand{\nomeTutorInterno}{Ombretta}
\newcommand{\cognomeTutorInterno}{Gaggi}

\newcommand{\prospettoSettimanale}{
     % Personalizzare indicando in lista, i vari task settimana per settimana
     % sostituire a XX il totale ore della settimana
    \begin{itemize}
        \item \textbf{Prima Settimana (40 ore)}
        \begin{itemize}
            \item Incontro conoscitivo con i vari membri dell'azienda;
            \item Studio delle basi dello stack tecnologico utilizzato dall'azienda;
            \item Studio dei metodi di sviluppo utilizzati in azienda;
        \end{itemize}
        \item \textbf{Seconda Settimana - Studio di Azure (40 ore)} 
        \begin{itemize}
            \item ;
        \end{itemize}
        \item \textbf{Terza Settimana - Sviluppo (40 ore)} 
        \begin{itemize}
            \item ;
        \end{itemize}
        \item \textbf{Quarta Settimana - Sottotitolo (XX ore)} 
        \begin{itemize}
            \item ;
        \end{itemize}
        \item \textbf{Quinta Settimana - Sottotitolo (XX ore)} 
        \begin{itemize}
            \item ;
        \end{itemize}
        \item \textbf{Sesta Settimana - Sottotitolo (XX ore)} 
        \begin{itemize}
            \item ;
        \end{itemize}
        \item \textbf{Settima Settimana - Sottotitolo (XX ore)} 
        \begin{itemize}
            \item ;
        \end{itemize}
        \item \textbf{Ottava Settimana - Conclusione (XX ore)} 
        \begin{itemize}
            \item ;
        \end{itemize}
    \end{itemize}
}

% Indicare il totale complessivo (deve essere compreso tra le 300 e le 320 ore)
\newcommand{\totaleOre}{}

\newcommand{\obiettiviObbligatori}{
	 \item \underline{\textit{O01}}: Realizzazione POC;
	 \item \underline{\textit{O02}}: Simulazione di un utilizzo massivo dell'applicazione;
	 \item \underline{\textit{O03}}: Realizzazione documentazione;
	 
}

\newcommand{\obiettiviDesiderabili}{
	 \item \underline{\textit{D01}}: Analisi approfondita e comparazione di diverse architetture e sul loro impatto sui costi;
	 \item \underline{\textit{D02}}: Analisi dei competitor per osservare come sono stati gestiti i problemi di carico ;
}

\newcommand{\obiettiviFacoltativi}{
	 \item \underline{\textit{F01}}: Analisi sviluppi futuri;
	 \item \underline{\textit{F02}}: Realizzazione di una prima versione dell'interfaccia utente, non solo funzionale ma anche conforme agli standard qualitativi di un'applicazione moderna, che possa essere la base della futura UI;
}